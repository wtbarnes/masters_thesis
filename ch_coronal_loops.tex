\chapter{Coronal Loops}
\label{ch:coronal_loops}
\hl{This chapter will discuss the discrete nature of corona in terms of coronal loop structures; Need a section on general plasma dynamics of loops to discuss energy transfer/loss/gain through heating/enthalpy/radiation/draining/fillingl; Also discuss general structure and how they are formed ;Give some general characteristics about them like length, temperature, density, through what layers they extend etc.; Show nice schematic}
\par As discussed in \S\ref{subsec:dynamo_flux}, magnetic field lines emerge from below the photosphere due to magnetic bouyancy and differential rotation, extending high into the atmosphere with their footpoints rooted in the solar surface. These approximately semi-circular structures are called \textit{coronal loops} and they are considered the primary building blocks of the solar corona \citep{reale_coronal_2010}. \hl{some more stuff about loops emitting in the x-ray, brief history, RTV papers, not more than 1-2 paragraphs}
\section{Observations}
\label{sec:observations}
\hl{Discuss some observations of loops and what has been learned about them, what constraints, multi-stranded versus single stranded
Show some pretty pictures}
\section{Modeling}
\label{sec:modeling}
\hl{Discuss modeling approaches, hydrodynamics versus magnetohydrodynamics, etc.}
\subsection{Magnetohydrodynamics}
\subsection{Hydrodynamics}
%?\subsection{Kinetic Modeling}
\section{Summary}
